\chapter{Discussion}
Invariant in time, 
macroscopical, directly observable level.

but adding the term “net” for macroscopic descriptions will usually
eliminate any ambiguity.

Distinguishing between the macroscopic and molecular viewpoints, as we have been
doing, is useful, often necessary, in science. Whether we consider the atmosphere as a
whole, a cloud, or a single water droplet, each system can be studied either as a macroscopic
entity (with our senses or suitable instrumentation), alternatively as a collection of
individual molecules that interact with each other in diverse and complex ways

Thermodynamics, that branch of science concerned with energy in its various forms,
regards nature from the macroscopic point of view.

The discipline of statistical mechanics, however, allows us to make connections
between the actions of the molecules in a system and its macroscopic, thermodynamic
properties.

changes in state of matter.

but recognize that the
conclusions we draw from this idealized example will apply equally well to


If we algebraically regroup the variables by

variation of the clausius clayperon eqaution applied to atmospheric conditions.

impose constraints on the relative behavior of the variables.

The system cannot violate the principle of energy conservation, for instance.

Clausius–Clapeyron: des /dT, gives the rate at which the vapor pressure increases with temperature

% above is copied
Historically cloud specific parameterization have been developed. The categorisation into cloud regimes have lead to problems. There is a urgent need for general purpose parametrization of clouds. Its believed that focusing on fewer variables stands a better chance of reducing the uncertainty associated with modelling cloud processes. Employing a hierarchical modelling approach, focusing on macrophysical properties.

Possible extension: Another possibility as \cite{Fowler1996LiquidAssumptions} mentions, is to parameterizing microphysical properties based on the \acrshort{cfc}.

\section{Artefact}
To årsaker til at man sannsynligvis får mer med skyer der det er aersoler,  ... finn i gullboka
\begin{enumerate}
    \item en
    \item to
\end{enumerate}


\section{Other}
Its worth noting that the satellite data using in this thesis is not included in  \citeauthor{Stubenrauch2013AssessmentPanel} study, but its illustrates nicely the large differences among other datasets. 


Siden det er en passive sendfor kan man forvente at cloud masks blir overestiment jo mer off nadir man beveger seg pga. strålen går lengder det det blir et større område å treffe sky eller aerosol på. \citeauthor{Maddux2010ViewingProducts}


It can be worth noting that the all sky radiance's from \acrfull{msg} in the period 2003-2012 is included in the assimilation. This is the same satellite that provides the cloud mask. 
\textbf{One sentence explaining that we have judged the appropriate use and have come to the conclusion that using the assimilated variables and not parameterised are appropriate for this application.}

\section{Notes}
\begin{enumerate}
    \item Ar - modeller er uavhengige modeller, hvor naboer er trent på korrelert data, uten informasjon er
    \item Nabolag rundt transformasjonen. 
    \item The distribution of values
    by feature is shown in fig. 2.
    \item Create correlation matrix using 
    \item We will keep in mind both the scores of 0.743 and 0.782, serving
    as fair baselines for a very simple, and slightly more sophisticated
    regression fit, respectively.
    \item Plotting R2 vs epoch, set ylim to 0,1. Not interresting to see where it learns.
    \item Summaries the best five architecture 
    \item A drawback of introducing more layers
    is that it increases the complexity, and thereby the chance of
    \item emphesizing a good agreement for latitudinal variation.
\end{enumerate}

\section{Good phases}
\begin{itemize}
    \item  In its most simple form, the diffusion equation is given by
    \item By using a finer grid one can usually get better approximations
    \item The comparison will focus on computational time and accuracy.
    \item This was chosen after
studying the development of the loss function as a function of number of iterations
    \item and is mandatory in
more advanced architectures(e.g. residual nets) where a constant spatial
dimension is demanded.
    \item a list of examples and so forth.
    \item (I will reference to source code/project part where relevant!)
    \item out of sample precision.
    \item Learn how to create plots with a zoomed in view.
    \item sufficiently large
    \item Viktig poeng. \textit{However, both academic
researchers and practitioners alike acknowledge the
need to make tests on the actual data set that is
subject of interest, as well as dedicating time and
resources to tune hyper parameters}
    \item methods for tabular data vs images
    \item I have opted to use
    \item We have also modified our own code for a dense feed forward neural
network produced for Project 2, see
    \item Gradient methods are at the heat of every machine learning algorithms. 
\end{itemize}

\begin{enumerate}
    \item Hvorfor det er viktig - prøv uten dommedag vri.
    \item Hva du vil gjøre for å løse det -  machine læring.
    \item Hva har andre gjort for å løses det - traditionell probabilistic modelling. 
    \item Sometimes emissions of greenhouse gases are referred to as climate forcers.
    \item Forskjell på earth system models, ECM and global circulation models, GCM is that ECM includes a carbon cycle. Which is very important when studying the carbon emissions of different RCPs and SSPs.
\end{enumerate}

\begin{enumerate}
    \item How representative is the training period we choose?
    \item testing (in-sample error) and validiation (out of sample error) - generalization error 
    \item Look at the correlation in the data. Pearson correlation? Someone else correlation?
    \item \textit{This plot reveals several things. First, the correlation is very strong. Second, ... }
    \item \textit{You will often gain good insight on the problem by examining data.}
    \item Ensamble methods in climate models and machine learning. It true that for both domains the model mean usually outperform the single model.
\end{enumerate}

\section{Future work}
In future work it would be interesting to asses how data driven parametrisation compare to the existing parametrisaions available in the state of the art climate models. Here both the temporal and spatial resolution is a lot coarser. Other data sets could be considered. The masks in other data sets are computed based on more channels than in METEOSAT but the temporal resolution is a lot worse. 


\textbf{From the introduction.}

%Logistic regression and Naive Bayes classifiers are exampls of algorithmns that predates computers, but are still very useful to this day. 
Internet continuous to provide large amounts of data from Wikipedia, Flicker (tagged images) and YouTube. Technological advances such as the \acrfull{gpu} allow for fast computations. It was originally developed for gaming. The invention of the interface CUDA allows performing heavy computations. This did wonders for machine learning.
%Advances in computational powers, such as graphical processing units, GPU's \textit{provide a environment/platform=os to learn in/on}. These where originally develop for the gaming industry, but in 2007 they realised a interface called CUDA (2007) which allows for computing \textbf{find a up to date cost and flops (floating point operations per second)}. \textbf{siter Chollet bok}
\\ \\ 
For clarity, the deep in deep learning refer to the number of layers. Moving from shallow networks to deeper ones (more than 10) algorithmic advances in gradient propagation was needed. The main advantage of using deep neural networks is that they find their own feature representations of the given data. \acrfull{ai} promises that if you have enough data you can find any relation. However this is not always the case. Often you have noise, non stationary system (e.g. climate) and/or miss-labeled data. The last  being a consequence of humans labelling data. Earth system monitoring provides a global view of variables across meteorological systems. %\textbf{Some thing about satellite era}. These large amounts of data and the flexible nature of the neural network makes is a suitable method also in geosciences. \textbf{With enough data neural networks can serve as a universal function approximate given a suitable hyper parameter tuning and input data.} 




