\begin{figure*}[h] % h means place here if possible
        \centering
        \begin{subfigure}[b]{\textwidth}   
            \centering 
            
\begin{tikzpicture}[ % GLOBAL CFG
    font=\sf \scriptsize,
    >=LaTeX,
    scale = 0.89,
    every node/.style={scale=0.89},
    % Styles
    cell/.style={% For the main box
        rectangle, 
        rounded corners=5mm, 
        draw,
        very thick,
        },
    operator/.style={%For operators like +  and  x
        circle,
        draw,
        inner sep=-0.5pt,
        minimum height =.70cm,
        },
    function/.style={%For functions
        ellipse,
        draw,
        inner sep=1pt
        },
    ct/.style={% For external inputs and outputs
        circle,
        draw,
        line width = .75pt,
        minimum width=1cm,
        inner sep=1pt,
        },
    gt/.style={% For internal inputs
        rectangle,
        draw,
        minimum width=12mm,
        minimum height=7mm,
        inner sep=1pt
        },
    mylabel/.style={% something new that I have learned
        font=\scriptsize\sffamily ,
        opacity = 0.2, 
        size = \large,
        },
    ArrowC1/.style={% Arrows with rounded corners
        rounded corners=10cm,
        thick,
        },
    ArrowC2/.style={% Arrows with big rounded corners
        rounded corners=.5cm,
        thick,
        },
    ]
    
    \node [gt, fill = yellow, opacity = 1.0, label = {\large Neural network  layer}] (ibox4) at (-2.75, 0) {}; 
    \node [operator, fill = pink, opacity = 1.0, label = {\large Pointwise operation}] (mux1) at (1, 0) { }; 
    
    % Vector transfer element
    \node [operator, fill = pink, opacity = .0] (n1) at (3.5, 0) { }; 
    \node [operator, fill = pink, opacity = .0] (n2) at (6.5, 0) { }; 
    \draw [->, ArrowC2, opacity = 1.0] (n1) -- (n2) node[midway, above=3.5mm of n1] {\large Vector transfer};
    
    % copy
    \node [operator, fill = pink, opacity = .0] (c1) at (10, 1.1) { }; 
    \node [operator, fill = pink, opacity = .0] (c2) at (10, -1.1) { };
    \node [operator, fill = pink, opacity = .0] (c3) at (8, 0) { }; 
    
    \draw [->, ArrowC2, opacity = 1.0] (c3 -| c1)++(-2., 0) -| (c1) node[above=-0.5mm of c3] {\large Copy};
    \draw [->, ArrowC2, opacity = 1.0] (c3 -| c2)++(-2., 0) -| (c2) ;

    % Concatenate
    \node [operator, fill = pink, opacity = .0] (co1) at (10, 1.0) { }; 
    \node [operator, fill = pink, opacity = .0] (co2) at (10, -1.0) { };
    \node [operator, fill = pink, opacity = .0] (co3) at (12, 0) { }; 
    
    %\draw [->, ArrowC2, opacity = 1.0, label = {Vector Transfer}] (c1) -- (c3) ;
    %\draw [->, ArrowC2, opacity = 1.0, label = {Vector Transfer}] (c1 -| c3)++(.5, 3.5) -| (c1) ;

    %\draw [->, ArrowC2, opacity = 1.0] (co1) |- (co3); %(co1) -- (co3);
    %\draw [->, ArrowC2, opacity = 1.0] (co2) |- (co3) node[above=0.5mm of co3] {\large Concatenate};
    
    \end{tikzpicture}
    
    
    


            \caption{Legend for LSTM walk through.}
            \label{fig:legend_lstm2}
        \end{subfigure}

        \begin{subfigure}[b]{0.475\textwidth}
            \centering
            \input{Chapter3_Method/TiKZ/lstm.tikz}
            \caption[Channel WV 6.2]
            {{\small Recurrent unit. LSTM cell.}}
            \label{fig:lstm_unit}
        \end{subfigure}
        \hfill
        \begin{subfigure}[b]{0.475\textwidth}  
            \centering 
            \input{Chapter3_Method/TiKZ/1cell_state_core_idea.tikz}
            \caption[]%
            {{\small The cell state. Keeping track of the long term dependencies.}}    
            \label{fig:cell_state_information_flow}
        \end{subfigure}
        \caption{Walk trough the components of a LSTM and the relevant equations.  Modified version of the example given at \cite{lstm_cell_tikz} and inspired by \cite{colah_blog_post}.
        }
        \label{fig:LSTM2}
    \end{figure*}