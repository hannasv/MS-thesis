\begin{figure}[h] % h means place here if possible
    \centering
        \begin{tikzpicture}[
    % GLOBAL CFG
    font=\sf \scriptsize,
    >=LaTeX,
    % Styles
    cell/.style={% For the main box
        rectangle, 
        rounded corners=5mm, 
        draw,
        very thick,
        },
    operator/.style={%For operators like +  and  x
        circle,
        draw,
        inner sep=-0.5pt,
        minimum height =.2cm,
        },
    function/.style={%For functions
        ellipse,
        draw,
        inner sep=1pt
        },
    ct/.style={% For external inputs and outputs
        circle,
        draw,
        line width = .75pt,
        minimum width=1cm,
        inner sep=1pt,
        },
    gt/.style={% For internal inputs
        rectangle,
        draw,
        minimum width=4mm,
        minimum height=3mm,
        inner sep=1pt
        },
    mylabel/.style={% something new that I have learned
        font=\scriptsize\sffamily
        },
    ArrowC1/.style={% Arrows with rounded corners
        rounded corners=.25cm,
        thick,
        },
    ArrowC2/.style={% Arrows with big rounded corners
        rounded corners=.5cm,
        thick,
        },
    ]

%Start drawing the thing...    
    % Draw the cell: 
    \node [cell, minimum height =1.5cm, minimum width=2cm, fill=cyan!50] (first) at (-1.0, 0){\Large \textbf{A}}; 
    \node [cell, minimum height =1.5cm, minimum width=2cm, fill=cyan!50] (second) at (2.5, 0){\Large \textbf{A}};
    \node [cell, minimum height =1.5cm, minimum width=2cm, fill=cyan!50] (third) at (6,0){\Large \textbf{A}};
    \node [cell, minimum height =1.5cm, minimum width=2cm, fill=cyan!50] (fourth) at (11,0){\Large \textbf{A}};

% Start connecting all.
    %Intersections and displacements are used. 
    % Drawing arrows    
    %\draw [->, ArrowC1] (first) -- (second);
    %\draw [->, ArrowC1] (second) -- (third);
    %\draw [->, ArrowC1] (third) -- (fourth);
    %\draw [->, ArrowC1] (first) -- (second);

    %\node[ct, label={[mylabel]Cell state}] (c) at (-4,1.5) {\empt{c}{t-1}};
    \node[ct, label={[mylabel]Output}, fill = red!50] (h) at (-1, 2) {\large $h_{0}$}; % , fill=blue
    \node[ct, label={[mylabel]below:Input}, fill = green!50] (x) at (-1, -2) {\large $x_0$}; %, fill = magenta
    \draw [->, ArrowC1] (x) -- (first);
    \draw [->, ArrowC1] (first) -- (h);

    %\draw [->, ArrowC1] (first -| first)++(1.5,0) -| (first); 
    %\draw [->, ArrowC1] (h -| ht)++(-0.5,0) -| (ht);
    %\draw [->, ArrowC1] (h -| ht)++(-0.5,0) -| (ht);
    %\draw [->, ArrowC1] (h -| ht)++(-0.5,0) -| (ht);
    
    \node[ct, label={[mylabel]Output}, fill = red!50] (h2) at (2.5, 2) {\large $y_{1}$};
    \node[ct, label={[mylabel]below:Input}, fill = green!50] (x2) at (2.5, -2) {\large $x_1$};
    \draw [->, ArrowC1] (x2) -- (second);
    \draw [->, ArrowC1] (second) -- (h2);
    
    \node[ct, label={[mylabel]Output},  fill = red!50] (h3) at (6, 2) {\large $y_{2}$};
    \node[ct, label={[mylabel]below:Input}, fill = green!50] (x3) at (6, -2) {\large $x_2$};
    \draw [->, ArrowC1] (x3) -- (third);
    \draw [->, ArrowC1] (third) -- (h3);
    
    \node[ct, label={[mylabel]Output},  fill = red!50] (ht) at (11, 2) {\large $y_{t}$};
    \node[ct, label={[mylabel]below:Input}, fill = green!50] (xt) at (11 , -2) {\large $x_t$};
    \draw [->, ArrowC1] (xt) -- (fourth);
    \draw [->, ArrowC1] (fourth) -- (ht);  
    
    \path[->, thick, black] (first) edge [out=90, in=180] node[above, midway] {\Large $h_0$} (second) ;
    \path[->, thick, black] (second) edge [out=90, in=180] node[above, midway] {\Large $h_1$} (third);
    \path[->, thick, gray, dashed] (third) edge [out=90, in=180] node[above, midway] {} (fourth);
    %\path[->, thick, black] (third) edge [out=90, in=180] (fourth);
    
    %\draw (first) to [out=0, in=0,looseness=8] (first);
    \end{tikzpicture}
    
    \caption{Unrolling Figure \ref{fig:rnn} in time yields this structure. Inspired by \cite{colah_blog_post}. % Used code from lstm unit to develop this one.
    }
    \label{fig:rnn_unrolled}
\end{figure}