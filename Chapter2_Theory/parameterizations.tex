\section{Parametrization of clouds} \label{sec:param_clouds}
%\textit{In doing so, we do not include the effects of changes to the cloud microphysics explicitly.} Building a model based on meteorological variables provided by a reliable estimate from reanalysis datasets.
All global climate simulations are limited by computational power and the typical model grids are much too coarse to resolve all relevant processes governing clouds. Parameterization allows us to nevertheless simulate the effects of clouds on the climate, through simplified representations of cloud processes that are a function of resolved model variables. The development of both observational and modelling systems requires a understanding of the physical and biogeochemical processes that take place in the earth system (\cite{Simmons2016Observation2016-2025}). New ideas are implemented into models and tested. The simulated climate should be in accordance with observations and be able to recreate previous climate. 

The complex nature of clouds originates from lots of different processes occurring simultaneously on different scales. Incorporating all these interactions into a model framework has proven to be difficult (\cite{IPCC_CH9_climate_models}, \cite{IPCC_CH7_clouds} +++ ). \textbf{Finn multiple sources} 

% https://journals.ametsoc.org/doi/pdf/10.1175/2009JAS3072.1
% https://www.esrl.noaa.gov/psd/iasoa/vocabulary/Cloud%20Properties/Macrophysical
% Due to small scale of the processes involved, modelling the effects of clouds on the earth system requires parameterizations. 
% Due to the large uncertainty in \acrfull{ecs} related to clouds. 
%This area has received a lot of attention the last few years. A consequence of increasing the complexity of the models is the trailing increase in uncertainty. Popular approaches are saturation threshold, \acrfull{pdf} and \acrfull{crm}. 
%The data driven approach taken in this thesis does not net this case-specific adjustment. Doesn't need different approaches for different regimes, but relies on the satellites capabilities to detect them. Cirrus being the most difficult. When you one as seperate parameterizations for different cloud regimens, a consequence of this is that the cloud fraction might exceed 1.

\subsection{Relative humidity schemes}
The simplest form of cloud scheme is a binary. A model grid box is either cloudy or clear. Equation \eqref{eq:binary_param_clouds} describes a diagnostic relationship between cloud cover and relative humidity. Binary saturation threshold can be implemented as follows,
\begin{equation} \label{eq:binary_param_clouds}
    CFC\left(RH\right) = 
     \begin{cases}
       \text{0,} &\quad\text{if RH}\le100\\
       \text{1,} &\quad\text{else}
     \end{cases}
\end{equation}

\subsection{Statistical schemes}
Most climate models have a fractional cloud cover, which is driven by a saturation threshold. All the vapour in excess of this threshold, often $RH=100\%$, gets transformed into cloud liquid water. A representation of sub-grid scale variability is necessary to achieve fractional cloud cover. The most common variables either alone or in combination are relative humidity, temperature and vertical velocity (\cite{Golaz2002_part1}, ++ ).\textbf{cite more papers parametrizing this - draw inspiration from table?}  

Based on observations from airplane campaigns, researchers have attempted to draw statistical distributions of relevant variables. These \acrfull{pdf} are implemented into models. Virtually all existing \acrshort{pdf}s have been used to model either cloud cover or its dependent variables. % humidity, temperature and so on. 
%TS: Hva prøver du egentlig å si her over ("Virtually all...")? Jeg henger ikke helt med i resonnementet
A reproduction of the summary from \cite{Tompkins2009CloudParametrization}, describing distribution used in a selection of papers is given in Table \ref{tab:summary_PDF}.
\begin{table}[ht]
    \centering
    \setlength\tabcolsep{1.5pt} % default value: 6pt
    \setlength\extrarowheight{-7pt}
    \begin{tabular}{c|c|c}
        PDF shape &  Summary & Reference \\ \hline
        Double Delta & U, S & Ose (1993), Fowler et al. (1996) \\
        Uniform & U, S & LeTreut and Li (1991) \\
        Triangular & U, S & Smith (1990), Rotstayn (1997), Nishizawa (2000) \\
        Polynomial & U, S & Lohmann et al. (1999) \\
        Gaussian & U, S & Bougeault (1981), Ricard and Royer (1993) \\ 
        & &  Bechtolf et al. (1995) \\
        Beta & U, sk & Tomkins (2002) \\
        Log-normal & U, sk & Bony and Emanuel (2001) \\ 
        Exponential &  U, sk & Bougeault (1981), Ricard and Royer (1993) \\
        & &  Bechtolf et al. (1995) \\
        Double Gaussian/ Normal & B, sk & Lewellen and Yoh (1993), Golaz et al. (2002)
    \end{tabular}
    \caption{Reproduction of summary in \cite{Tompkins2009CloudParametrization}. Distributions used to parameterize cloud or its dependant variables. The key to decipher the summary column; U=Unimodal, B=Bimodal, S=Symmetric, sk = Skewed.}
    \label{tab:summary_PDF}
\end{table}
Researchers have not been successful in finding an adequate representation of cloud cover using these approaches (\cite{Tompkins2009CloudParametrization}). 

\cite{Golaz2002_part1} derived a joint \acrshort{pdf} of the sub-gridscale variability, serving as the base for parameterizing boundary layer clouds. This scheme is implemented in \acrfull{noresm} (\cite{SelandNORESM}) and \acrfull{cesm}, to recognised \acrshort{esm} (\cite{DanabasogluCESM}).
%TS: For NorESM kan du referere til Seland et al. (2002), og for CESM2 kan du referere til Danabasoglu et al. (2020)
The parameterization can be considered a higher-order turbulent closure problem. The first (mean), second (variance) and third order statistical moments, of the vertical velocity ($w$), the liquid water potential temperature ($\theta_l$), and the total water specific humidity ($q_t$) determines the family of \acrshort{pdf}s. It is designed to be flexible enough to (avoid the use of) circumvent the case specific adjustment. For more details see \cite{Golaz2002_part1} and \cite{Golaz2002_part2}.

%What is necessary to understand why clouds are Parameterisations. Cite that all climate models are wrong but some are useful.
\subsection{Cloud resolving models} \label{sec:params_climate_models}
Another method of cloud parameterization is using \acrfull{crm} that are integrated into global climate models. Contrary to what the name implies, this type of model still has problems with resolving the very smallest cloud processes, occurring on micrometer-scales. 

Running an \acrfull{les}-model, the increased resolution is able to resolve convective motions, but microphysical processes and turbulence effects still require parameterizations. \citeauthor{Baba2019SpectralModel} used this approach for parameterizing a spectral cumulus cloud. Obtaining the entrainment rate based on cloud properties from a \acrshort{crm} they built a parameterization valid for both shallow and deep convection. \textit{Entrainment rate} is the rate at which surrounding air penetrates the cloud. Preserving the physical properties, it is important that the method handles co-existing phenomena (\cite{Baba2019SpectralModel}). 

%Accelerating the speed of computations is always useful. 
\acrshort{dl} provides suitable methods for emulating %emulating what?, 
aimed to accelerate the speed of the heavy computations in \acrshort{crm}. Emulation in the context of computing refers to imitation of one model using another one. In this example the statistical models are used to mimic the behavior of the physically based \acrshort{crm}. The \acrshort{dl} model performance is restricted by the  \acrshort{crm}, performing at best as good as the \acrshort{crm} (\cite{Rasp2018DeepModels}).