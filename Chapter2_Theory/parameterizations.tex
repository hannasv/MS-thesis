\section{Parametrization of clouds} \label{sec:param_clouds}
%\textit{In doing so, we do not include the effects of changes to the cloud microphysics explicitly.} Building a model based on meteorological variables provided by a reliable estimate from reanalysis datasets.
All global climate simulations are limited by computational power and the typical model grids are much too coarse to resolve all relevant processes governing clouds. Parameterization allows us to nevertheless simulate the effects of clouds on the climate, through simplified representations of cloud processes that are a function of resolved model variables. The development of both observational and modelling systems requires a understanding of the physical and biogeochemical processes that take place in the earth system (\cite{Simmons2016Observation2016-2025}). New ideas are implemented into models and tested against observations. The simulations should be able to recreate previous climate. 

The complex nature of clouds originates from lots of different processes occurring simultaneously on different scales. Incorporating all these interactions into a model framework has proven to be difficult (\cite{IPCC_CH7_clouds}, pp. 584)
 
%This area has received a lot of attention the last few years. A consequence of increasing the complexity of the models is the trailing increase in uncertainty. Popular approaches are saturation threshold, \acrfull{pdf} and \acrfull{crm}. 

In the literature is common to distinguish/\textbf{discuss} between prognostic and diagnostic variables/\textbf{schemes}. 
A prognostic variable predicts the values for other variables in the future, while a disagnostic variable is time-independant or links values they have on identical times.

\subsection{Relative Humidity and Statistical Schemes}
The simplest form of cloud scheme is a binary. A model grid box is either cloudy or clear. Equation \eqref{eq:binary_param_clouds} describes a diagnostic relationship between cloud cover and relative humidity. Binary saturation threshold can be implemented as follows,
\begin{equation} \label{eq:binary_param_clouds}
    CFC\left(RH\right) = 
     \begin{cases}
       \text{0,} &\quad\text{if RH}\le100\\
       \text{1,} &\quad\text{else}
     \end{cases}
\end{equation}
Many high-resolution model apply this appoach, but it is clearly not suitable for \acrshort{esm} having spatial resolution in the scale of 100km (\cite{Tomkins2005}).

Most climate models have a fractional cloud cover, which is driven by a saturation threshold. All the vapour in excess of this threshold, often $RH=100\%$, gets transformed into cloud liquid water. An assumption of sub-grid scale variability is necessary to achieve fractional cloud cover. %The most common variable is relative huimdity, this often appear in combination with 
The most common variables either alone or in combination are relative humidity, temperature and vertical velocity (\cite{Golaz2002_part1}). A necessary, but rough simplification applied is a fixed threshold for the critical relative humidity. 

In statistical schemes relative humidity and other dependant variables are simulated using \acrlong{pdf}s. Such distribution are difficult to obtain theoretically and a common approach is to draw these distributions empirically, based on observations to airplane campaigns. In this way the functional form of the underlying distributions have a physical basis. Observations have been made during varying cloud conditions and almost all existing \acrshort{pdf}s have been used in statistical schemes. The parameterization is then very sensitive to the choice of moments, i.e. mean, variance, skewness and kurtoisis. It can be shown that when employing fixed moments in statistical scheme some can be reduced to RH-schemes (\cite{Tomkins2005}). Researchers have not been successful in finding an adequate representation of cloud cover using these approaches (\cite{Tompkins2009CloudParametrization}). 

%\acrshort{pdf}s of unbounded functions can casue issues when representing maximumd cloud condensate mixing ratio, as it approach infinity a grid box will always be covered in cloud. 
%This can be problematic with unbounded functions Lognormal, Gamma, Gaussian and Exponential. For example if the maximum cloud condensate mixing ration

\citepaper{Golaz2002_part1} derived a joint \acrshort{pdf} of the sub-gridscale variability, serving as the base for parameterizing boundary layer clouds. This scheme is implemented in \acrfull{noresm} (\cite{SelandNORESM}) and \acrfull{cesm}, to recognised \acrshort{esm} (\cite{DanabasogluCESM}). The parameterization can be considered a higher-order turbulent closure problem. The first (mean), second (variance) and third order statistical moments, of the vertical velocity ($w$), the liquid water potential temperature ($\theta_l$), and the total water specific humidity ($q_t$) determines the family of \acrshort{pdf}s. It is designed to be flexible enough to %(avoid the use of) 
circumvent the case specific adjustment (\cite{Golaz2002_part1},  \cite{Golaz2002_part2}). 

\subsection{Cloud resolving models} \label{sec:params_climate_models}
Another method of cloud parameterization is using \acrfull{crm} that are integrated into global climate models. Contrary to what the name implies, this type of model still has problems with resolving the very smallest cloud processes, occurring on micrometer-scales. 
\acrshort{crm} are computationally expensive and can only be run for a short amount of time. One weakness by drawing (empirical) relationships from \acrshort{crm}-models is that you rely on their parametrizations of microphics for instance (\cite{Tomkins2005}).

%Running an \acrfull{les}-model, the increased resolution is able to resolve convective motions, but microphysical processes and turbulence effects still require parameterizations. \citeauthor{Baba2019SpectralModel} used this approach for parameterizing in cloud properties cumulus clouds handling both shallow and deep convection. Obtaining the entrainment rate based on cloud properties from a \acrshort{crm} they built a parameterization valid for both shallow and deep convection. \textit{Entrainment rate} is the rate at which surrounding air penetrates the cloud.\textbf{REWRITE. Preserving the physical properties, it is important that the method handles co-existing phenomena (\cite{Baba2019SpectralModel}). }
% TS : skjønner ikke poenget. 

%Accelerating the speed of computations is always useful. 
\acrshort{dl} provides suitable methods for emulating \textbf{find a example where a emultor is used to speed up computations, weather prediction, regional model?} 
%TS emulating what?, 
aimed to accelerate the speed of the heavy computations in \acrshort{crm}. Emulation in the context of computing refers to imitation of one model using another one. In this example the statistical models are used to mimic the behaviour of the physically based \acrshort{crm}. The \acrshort{dl} model performance is restricted by the  \acrshort{crm}, performing at best as good as the \acrshort{crm} (\cite{Rasp2018DeepModels}).

\subsection{ECMWF IFS model} \label{sec:era5_param}
%\textbf{Read Forbes in Downloads}
\acrfull{ecmwf}s numerical weather prediction model is named, Integrated Forecasting System (IFS), They frwuently update their model, this section is based on the Cycle 41r2. 

Revolutionary in its time \citepaper{Tiedtke1993} introdused a fully prognostic scheme for stratiform and convective clouds. Built on two variables cloud fractional cover and cloud condensate, seperated into liquid and snow by temperature. The scheme handles a comprehensive list of mechanisms, \textit{the scheme consderes the formation of clouds in connnection with large-scale acent, diabatic coooling, boundary layer turbuence and horizontal transport of cloud water from convective updrafts. Cloud dissipation through adiabatic and diabatic heating and turbulent mixing of cloud air with unsaturated envionmental air, the depletion of cloud water by pressipitation}. \textit{Allows for formation of anvils, cirrus clouds from convective updrafts and boundary layer clouds.} 
Except for some small modification, this was the operational scheme for 15 years (1995-2010). 

The representation of clouds by two variables lead to a number of simlpifications and in 2010 Forbes improved the scheme by extending the number of variables to 6 \textbf{?}
 \citepaper{Forbes2011AnPrecipitation} made some significant adjustments to precipitation advection, mixed phase clouds, 

Since then the IFS model has been upgraded with ice sedimentation and autoconversion of snow, along with subgrid precipitation and evaporation and ice supersaturation of cloud free air. 

