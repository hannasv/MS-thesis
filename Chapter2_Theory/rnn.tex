

\begin{figure}
    \centering
        \begin{tikzpicture}[
    % GLOBAL CFG
    font=\sf \scriptsize,
    >=LaTeX,
    % Styles
    cell/.style={% For the main box
        rectangle, 
        rounded corners=5mm, 
        draw,
        very thick,
        },
    operator/.style={%For operators like +  and  x
        circle,
        draw,
        inner sep=-0.5pt,
        minimum height =.2cm,
        },
    function/.style={%For functions
        ellipse,
        draw,
        inner sep=1pt
        },
    ct/.style={% For external inputs and outputs
        circle,
        draw,
        line width = .75pt,
        minimum width=1cm,
        inner sep=1pt,
        },
    gt/.style={% For internal inputs
        rectangle,
        draw,
        minimum width=4mm,
        minimum height=3mm,
        inner sep=1pt
        },
    mylabel/.style={% something new that I have learned
        font=\scriptsize\sffamily
        },
    ArrowC1/.style={% Arrows with rounded corners
        rounded corners=.25cm,
        thick,
        },
    ArrowC2/.style={% Arrows with big rounded corners
        rounded corners=.5cm,
        thick,
        },
    ]

%Start drawing the thing...    
    % Draw the cell: 
    \node [cell, minimum height =1.5cm, minimum width=2cm] (first) at (0,0){\Large \textbf{A}}; % , fill=green
    \node [cell, minimum height =1.5cm, minimum width=2cm] (second) at (3,0){\Large \textbf{A}};
    \node [cell, minimum height =1.5cm, minimum width=2cm] (third) at (6,0){\Large \textbf{A}};
    \node [cell, minimum height =1.5cm, minimum width=2cm] (fourth) at (10,0){\Large \textbf{A}};

% Start connecting all.
    %Intersections and displacements are used. 
    % Drawing arrows    
    \draw [->, ArrowC1] (first) -- (second);
    \draw [->, ArrowC1] (second) -- (third);
    \draw [->, ArrowC1] (third) -- (fourth);
    %\draw [->, ArrowC1] (first) -- (second);

    %\node[ct, label={[mylabel]Cell state}] (c) at (-4,1.5) {\empt{c}{t-1}};
    \node[ct, label={[mylabel]Hidden state}] (h) at (0, 2) {\large $h_{0}$}; % , fill=blue
    \node[ct, label={[mylabel]below:Input}] (x) at (0, -2) {\large $x_0$}; %, fill = magenta
    \draw [->, ArrowC1] (x) -- (first);
    \draw [->, ArrowC1] (first) -- (h);

    %\draw [->, ArrowC1] (first -| first)++(1.5,0) -| (first); 
    %\draw [->, ArrowC1] (h -| ht)++(-0.5,0) -| (ht);
    %\draw [->, ArrowC1] (h -| ht)++(-0.5,0) -| (ht);
    %\draw [->, ArrowC1] (h -| ht)++(-0.5,0) -| (ht);
    
    \node[ct, label={[mylabel]Hidden state}] (h2) at (3, 2) {\large $h_{1}$};
    \node[ct, label={[mylabel]below:Input}] (x2) at (3, -2) {\large $x_1$};
    \draw [->, ArrowC1] (x2) -- (second);
    \draw [->, ArrowC1] (second) -- (h2);
    
    \node[ct, label={[mylabel]Hidden state}] (h3) at (6, 2) {\large $h_{2}$};
    \node[ct, label={[mylabel]below:Input}] (x3) at (6, -2) {\large $x_2$};
    \draw [->, ArrowC1] (x3) -- (third);
    \draw [->, ArrowC1] (third) -- (h3);
    
    \node[ct, label={[mylabel]Hidden state}] (ht) at (10, 2) {\large $h_{t}$};
    \node[ct, label={[mylabel]below:Input}] (xt) at (10, -2) {\large $x_t$};
    \draw [->, ArrowC1] (xt) -- (fourth);
    \draw [->, ArrowC1] (fourth) -- (ht);    

    %\draw (first) to [out=0, in=0,looseness=8] (first);


\end{tikzpicture}
    
    
    \caption{Illustration of the flow of information in a one layer unrolled rnn. A is a recurrent unit. Inspired by http://colah.github.io/posts/2015-08-Understanding-LSTMs/. Consider removing the labels}
    \label{fig:rnn}
\end{figure}