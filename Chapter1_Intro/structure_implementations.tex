\section{Structure and Implementations}
The numerical methods used in this thesis are described in Chapter \ref{ch:num_methods}. All the code is available on GitHub in the project repository named MS on \href{https://github.com/hannasv/MS}{https://github.com/hannasv/MS}. 

In this repository you will find everything you need to perform this experiment yourself. It contains descriptions for downloading data this includes requesting the correct licences. Experiments are run in notebooks and project environments are provided to ease the reproduction of results. The project environment called ``sciclouds'' is ready for installation. This is a conda envionment, the yaml-file lists the python packages and the versions used for running this code. Below you find the code example for installing the project environment.

% Included in the readme file on github. 
\begin{verbatim}
git clone https://github.com/hannasv/MS.git
cd MS
conda env create -f environment.yml
conda activate sciclouds
python setup.py install
\end{verbatim}

Supplementary material for remapping satellite data and land-sea masks is available in the supplementary repository \href{https://github.com/hannasv/MS-suppl}{https://github.com/hannasv/MS-suppl}.

All the code is developed in Python3.7, some of the modules, especially the ``AR model'', draw inspiration from the structure of sklearn (\cite{sklearn_api}). The \acrshort{convlstm} is implemented in keras (\cite{chollet2015keras}) using tensorflow (2.0.0) as a backend (\cite{tensorflow2015}). The keras-tuner (1.0.0) is used to automize the hyperparameter search (\cite{chollet2015kerastuner})


Maps and other plots are generated the package \textit{Cartopy} (0.17) (\cite{Cartopy}). 

Illustrations are developed using TIkZ, a language used for producing technical illustrations within the environment of LaTEX.

Other packages to cite 
\begin{enumerate}
    \item pyaerocom 
    \item scipy for the psudo inverse
    \item xarray relieas heavily on pandas 
\end{enumerate}
