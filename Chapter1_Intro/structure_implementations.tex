\section{Structure and Implementations} \label{sec:structure_and_implementations}
The numerical methods used in this thesis are described in Chapter \ref{ch:num_methods}. The code is available on GitHub in the project repository named ``MS'' on \href{https://github.com/hannasv/MS}{https://github.com/hannasv/MS}. Instructions for downloading reanalysis (ERA5) data using python is provided.

The code is developed in Python3.7, a popular language for scientific software development. The source code is stored in the package \textit{sciclouds}, made available on Github trought the project repository. Developed modules draw inspiration from the structure of \textit{scikit-learn} (\cite{sklearn_api}) and the \textit{keras-tuner} (\cite{chollet2015kerastuner}). The \acrshort{convlstm} is implemented in \textit{keras} (\cite{chollet2015keras}) using \textit{tensorflow} as a backend (\cite{tensorflow2015}). The \textit{keras-tuner} is used to automize the hyperparameter search (\cite{chollet2015kerastuner}). 

Vizualizations are generated using \textit{Matplotlib} (\cite{matplotlib}), \textit{Seaborn} (\cite{seaborn}) and maps using the package \textit{Cartopy} (\cite{Cartopy}). Other illustrations are developed using TIkZ, a language used for producing technical illustrations within the environment of LaTEX.

%The repository contains everything need to reproduce the results in this study.
%The experiments are conducted in notebooks and the developed modules are stored in the package ``sciclouds''. Descriptions on how to acquire the data (scripts if possible) and project environment is provided to simplify the process.

The package versions are documented in the \textit{requirements.txt} and the project environment called ``sciclouds'' is ready for installation. This is a conda environment, the yaml-file lists the Python packages and requirements necessary for running this code. Below you find the code example for cloning the project and installing the environment.

% Included in the readme file on github. 
\begin{verbatim}
git clone https://github.com/hannasv/MS.git
cd MS
conda env create -f environment.yml
conda activate sciclouds
python setup.py install # installing package from source
\end{verbatim}

Supplementary material for remapping satellite data and filtering masks is available in the supplementary repository \href{https://github.com/hannasv/MS-suppl}{https://github.com/hannasv/MS-suppl}. %To make use of all the functionality available trought ``MS'', the supplementary repository needs to be cloned in the same directory.
The filters are generated from within the environment of PyAEROCOM (\cite{pyaerocom}). 


