\section{Structure and Implementations}
The numerical methods used in this thesis are described in Chapter \ref{ch:num_methods}. All the code is available on GitHub in the project repository named MS on \href{https://github.com/hannasv/MS}{https://github.com/hannasv/MS}. 

All the code is developed in Python3.7, some of the modules, especially the ``AR model'', draw inspiration from the structure of sklearn (\cite{sklearn_api}). The \acrshort{convlstm} is implemented in keras (\cite{chollet2015keras}) using tensorflow (2.0.0) as a backend (\cite{tensorflow2015}). 
The keras-tuner (1.0.0) is used to automize the hyperparameter search (\cite{chollet2015kerastuner}) and maps and other plots are generated using the package \textit{Cartopy} (0.17) (\cite{Cartopy}). 

The repository contains everything need to reproduce the results in this study. The experiments are conducted in notebooks and the developed modules are stored in the package ``sciclouds''. Descriptions on how to acquire the data (scripts if possible) and project environment is provided to simplify the process.

The project environment called ``sciclouds'' is ready for installation. This is a conda envionment, the yaml-file lists the python packages and the versions used for running this code. Below you find the code example for installing the project environment.

% Included in the readme file on github. 
\begin{verbatim}
git clone https://github.com/hannasv/MS.git
cd MS
conda env create -f environment.yml
conda activate sciclouds
python setup.py install
\end{verbatim}

Supplementary material for remapping satellite data and land-sea masks is available in the supplementary repository \href{https://github.com/hannasv/MS-suppl}{https://github.com/hannasv/MS-suppl}. The filters are generated from within the environment of PyAEROCOM (\cite{pyaerocom}).

Other more technical illustrations are developed using TIkZ, a language used for producing technical illustrations within the environment of LaTEX.
