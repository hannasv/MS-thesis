\chapter{Introduction} \label{ch:introduction}
%\textbf{The uncertain cloud feedback prevents us from predicting future climate with confidence - Geographycal distribution of cloud feedbacks}
Weather and climate have a major influence of life on Earth. While the globe struggles to support the needs of a increasing population, human activity continue to change the environment. %Emissions of greenhouse gases have a delayed influence on temperature. %For the current atmospheric composition, the surface temperature required to close the energy budget is higher than the global mean temperature of $255K$. \textbf{dette er -18 tempen jordkloden hadde hatt uten atmosphere}
% The \acrfull{ecs} is the amount of warming that will occur ones all 
Global warming is in progress. Greenhouse gas emission have a delayed effect on temperature and it will keep rising until it has reached the equilibrium temperature, closing the energy budget. 

Providing a realistic estimate of future climates is important for motivating mitigation and limiting the global temperature increase. 
% avoiding tipping points. 
This introduces the need for studying potential futures using a combination of models and idealised experiments. A common, yet not very intuitive quantity used to describe this temperature increase is the \acrfull{ecs}. 

Estimates based on %an ensemble of models, \acrfull{cmip5}, 
simulations prepared for the latest assessment report (\acrshort{ar5}) produced by the \acrfull{ipcc} puts \acrshort{ecs}s in the range from $2.1^oC$ to $4.7^oC$ (\cite{IPCC_CH9_climate_models}, pp.817). These runs are forced with instantaneous quadrupling the $CO_2$ concentration %(of pre-industrial levels 
and keeping it constant for 150 years. Then the \acrshort{ecs} is computed as the linear fit between perturbed global mean surface temperature and the radiative imbalance at \acrfull{toa}. 
%from runs forced with a 
\acrshort{ar5} was published in \citeyear{IPCC_entire_book}, since then a lot of research efforts have been invested in reducing the spread (\textbf{Trude suggestion citation}). The socio-economic consequences related to the uncertainty is enormous (\cite{bony2015}). 

% Despite decades of research efforts this uncertainty has been reduced from \textbf{mistenkter at perioden klimasensitiviten representerer er helt forskjellig.} % The first estimates of \acrshort{ecs} made by \cite{hansen} where in the range []. This is not major improvement based on the decades of research provided in this period. 

%\section{Scope of this study}
%Including the effects of clouds on large scale processes, in climate models, has proven difficult \textit{There is very high confidence that uncertainties in cloud processes explain much of the spread in modelled climate sensitivity.  } (\cite{IPCC_CH9_climate_models}). 

After decades of being mainly a research area of limited use, today \acrfull{ai} in the form of \acrfull{dl} has become a part of our daily life. Face recognition technology has the ability to unlock your phone. Speech recognition allows you to dictate text messages. % (\cite{Sak2014LongRecognition}).
Self-driving buses are used in metropolitan areas around the world, including Oslo city centre.  %(\href{https://www.letsholo.com/oslo}{https://www.letsholo.com/oslo}). 
Image manipulation allows you to transform Monet paintings to pictures and back (\cite{zhu2017_cycleGAN_monet_zebra}). Video manipulation can make people appear to say and do things they never did. 

\acrshort{dl} has already proven its value in many fields. The scope of this thesis is to implement and compare different methods for data driven learning to find the most suitable method for cloud cover predictions. For this task \acrfull{ar} models and \acrfull{convlstm} are considered. 
%\textit{This thesis attempts to answer if its a appropriate method for solving cloud predictions.} \textit{It remains a open question if its the best model for tackling this task, predicting cloud cover.} %Even thought machine leaning is has proven able for great advances in certain fields. 
Occam's razor is an old philosophical concept. It states that if two hypothesises are equally likely, the simplest one should be preferred. \textbf{kilde (ikke wikipedia)}. For this application, the compilation of a new dataset was deemed necessary. Named \acrfull{ecc}, and composed of reanalysis data from ERA5 and satellite retrievals from \acrfull{msg}. The dataset has a temporal resolution of one hour and spatial resolution of $0.25^o$. Clouds have a average lifetime of one hour (\cite{lohmann2016}), thus it should be achievable extract suitable features to make reasonable predictions. 

There is currently a huge interest in and activities around \acrfull{ml} and \acrshort{ai}. The practical applications to climate research are under investigation. The last few years researchers pursuit to incorporate \acrshort{dl} into geosciences. Attempting to solve a wide range of problems, from rainfall runoff modelling (\cite{hess-23-5089-2019}) to  high-resolution weather forecasting (\cite{Rodrigues2018DeepDownscale:Forecast}). Another more comprehensive \acrshort{ml} project is lead by Tapio Schneider at \acrfull{caltech}. Along with his team of technologist they have ambitions to create a \acrfull{esm} using \acrshort{ml}. He and his team of researchers from \acrfull{mit} and former employees of Microsoft and Google hope to create a platform that can resolve clouds and hopefully reduce the uncertainty in climate sensitivity (\cite{Voosen2018ScienceIntelligence}).

There is a high confidence that a lot if the spread is attributed to cloud feedback's (\cite{IPCC_CH9_climate_models}, pp.?). Developing new methods for cloud cover predictions is believed to produce more reliable estimates of future climates. Focusing on a small part of the processes, stands the best chance of reducing the spread. This thesis is a ``proof of concept'' study investigating the potential for using a data driven approach for parameterizing \acrfull{cfc} based on environmental variables. 
%at a satisfactory precision.

\section{Structure and Implementations}
The numerical methods used in this thesis are described in Chapter \ref{ch:num_methods}. The code is available on GitHub in the project repository named ``MS'' on \href{https://github.com/hannasv/MS}{https://github.com/hannasv/MS}. 


The code is developed in Python3.7, a popular language for scientific software development. The source code is stored in the package \textit{scloud}, made available on Github trought the project repository. Developed modules draw inspiration from the structure of \textit{scikit-learn} (\cite{sklearn_api}) and the \textit{keras-tuner} (\cite{chollet2015kerastuner}). The \acrshort{convlstm} is implemented in \textit{keras} (\cite{chollet2015keras}) using \textit{tensorflow} as a backend (\cite{tensorflow2015}). The \textit{keras-tuner} is used to automize the hyperparameter search (\cite{chollet2015kerastuner}).

Vizualizations are generated using \textit{Matplotlib} (\cite{matplotlib}), \textit{Seaborn} (\cite{seaborn}) and maps are generated using the package \textit{Cartopy} (\cite{Cartopy}). Other illustrations are developed using TIkZ, a language used for producing technical illustrations within the environment of LaTEX.

%The repository contains everything need to reproduce the results in this study.
%The experiments are conducted in notebooks and the developed modules are stored in the package ``sciclouds''. Descriptions on how to acquire the data (scripts if possible) and project environment is provided to simplify the process.
The versions are documented in the \textit{requirements.txt} and the project environment called ``sciclouds'' is ready for installation. This is a conda environment, the yaml-file lists the Python packages and requirements necessary for running this code. Below you find the code example for installing the environment.

% Included in the readme file on github. 
\begin{verbatim}
git clone https://github.com/hannasv/MS.git
cd MS
conda env create -f environment.yml
conda activate sciclouds
python setup.py install # installing package from source
\end{verbatim}

Supplementary material for remapping satellite data and filtering masks is available in the supplementary repository \href{https://github.com/hannasv/MS-suppl}{https://github.com/hannasv/MS-suppl}. The filters are generated from within the environment of PyAEROCOM (\cite{pyaerocom}). To make use of all the functionality this repo needs to be cloned in the same directory as ``MS''.



