\chapter{Introduction} \label{ch:introduction}
\section{Scope of this study}
Weather and climate have a major influence of life on earth. Global warming is now a fact and the temperature will keep rising until it has reached the equilibrium climate temperature. Emissions of greenhouse gases have a delayed influence on temperature. This introduces the need for studying potential futures using climate models and different sosio-economic pathways. Improved estimates of the temperature as a function of forcings is important for the motivating mitigation. By developing new methods for cloud cover predictions it may be possible to produce better estimates. These methods may also be useful in the fields of numerical weather prediction and cloud cover forecasting. The latter one is of importance for production of solar power.
% In January 2020 Mijødirektoratet published a report \textit{Klimakur 2030} describing \textbf{what?}. Main goal to avoid tipping points. 
\\ \\
%Climate models are a useful tool when studying climate change. 
Including the effects of clouds \textit{(on large scale processes)}, in climate models, has proven difficult. \textbf{kilde} This thesis will examine the potential for data driven methods to parameterise clouds. For this application, the compilation of a new dataset was deemed necessary. Named \acrfull{ecc}, and composed of reanalysis data from ERA5 and satellite retrievals from \acrfull{msg}. With a hourly temporal resolution and spatial resolution of $0.25^o$ it should be possible to extract suitable features to make predictions. Knowing that the average lifetime of a cloud is an hour \textbf{kilde Lohman}.
\\ \\ 
%Machine learning methods has shown great performance in other fields such as \textbf{XXX}. In this thesis \textbf{We/I} want to examine the potential of paramerising cloud cover using a machine learning and traditional statistical methods. 
\section{Why data driven learning} \label{sec:intro_deep_learning}
%The methods considered is traditional statistical models (Autoregressive models) and newer deep learning \textit{algorithms} (Convolutional LSTM). Recurrent networks have proven valuable in fields such as speech recognition and image captioning, but will the beat the more used, auto regressive models. 
You might not give it that much consideration, but today artificial intelligence is a part of our daily life. Face recognition technology have the ability to unlock your phone. Speech recognition allows you to dictate text messages. Self driving buses are present in the Oslo city centre and other towns. By manipulating images, you can force Putin to ride a horse. \textbf{cite GAN article} Video manipulation can make people say things they never did. Deep learning has already proven it self invaluable in many fields. This thesis intend to compare different methods for data driven learning to find the most suitable method for cloud predictions. For this task autoregressive models and \acrfull{convlstm} are considered. 
%\textit{This thesis attempts to answer if its a appropriate method for solving cloud predictions.} \textit{It remains a open question if its the best model for tackling this task, predicting cloud cover.} %Even thought machine leaning is has proven able for great advances in certain fields. 
Ockhams razor is a old philosophical concept. It states that if two hypothesis are just as likely, the simplest one is most probably correct. \textbf{kilde (ikke wikipedia)}.
\\ \\
There is a huge hype these days around machine learning and artificial intelligence.
The practical applications to climate research are under investigation. The last couple of years researchers have been attempting to incorporate machine learning into geosciences. Attempting to solve a wide range of problems, from rainfall runoff modelling (cite krazerts) to  high-resolution weather forecasting (cite Rodrigues).
% and air quality forecasting (sun and liu). %, precipitaiton nowcasting (Shi et al) and \textbf{kanskje: LES} \textit{deep neural network based feature representation for weather data.} \textbf{lui et al }. %Some attempts was more successful than others. 
Another more comprehensive machine learning project is lead by Tapio Schneider at Caltech. Along with his team of technologist they have ambitions to create a earth system model using machine learning. With his team of from MIT and former employees of Microsoft and Google they hope to create a platform which can resolve clouds and hopefully reduce the spread in climate sensitivity. \textbf{cite Science}
\\ \\
The data driven approach hold the potential to extract the correct information, across scales, necessary for predicting cloud cover at a satisfactory precision. Developing methods that eventually can contribute to reducing the spread in climate sensitivity. Producing a more accurate estimate of global mean temperature increase.
%\textit{With this in mind, the deep learning models will be compared to traditional statistical models.} %This thesis is concerned by comparing the advanced deep learning models to traditional statistical models (expanded versions with other predictions). 
%Hopefully it able to shed some light on improved estimates of cloud cover prediction. \textbf{En bedre setning på hva vi ønsker å bidra med.} 
%\textbf{Section on autoregresisve models - ikke skyt spurv med kanon}
%Det enkleste er ofte det beste, så det er viktig å huske å sammenligne med traditionelle statisktiske metoder. Her utvided litt til å være med en den traditionelle differens ligningen. 
% Statistical downscaling. Hugo article.
%\textit{Domain knowledge: On the one hand, you need someone with domain knowledge about the problem you are trying to solve. What are the possibilities and limitations within a system, and how can the solution be applied in practice?
%\\ \\ 
%Data Science: You also need the capabilities within Data Science, which involves everything related to analytics, statistics, signal processing, machine learning, artificial intelligence and deep learning. Essentially, methods and techniques to extract and utilize patterns and information in your data
%\\ \\
%Software engineering: Software engineering skills are crucial for building good data driven solutions. This involves setting up infrastructure for harvesting and processing data through proper pipelines and managing access to data, and in the end building functional and user-friendly software tools for the end users.}