\chapter{Introduction} \label{ch:introduction}
\section{Scope of this study}
Weather and climate have a major influence of life on earth. Global warming is now a fact and the temperature will keep rising until it has reached the equilibrium climate temperature. Emissions of greenhouse gases have a delayed influence on temperature. This introduces the need for studying potential futures using climate models and different sosio-economic pathways. Improving estimates of temperature based on human activity is a important aspect to motivate mitigation. %Improved estimates of the temperature as a function of forecings is important for the motivating mitigation. 
By developing new methods for cloud cover predictions it may be possible to produce better estimates. These methods may also be useful in the fields of numerical weather prediction and cloud cover forecasting. The latter one is of importance for production of solar power.
% In January 2020 Mijødirektoratet published a report \textit{Klimakur 2030} describing \textbf{what?}. Main goal to avoid tipping points. 

%Climate models are a useful tool when studying climate change. 
Including the effects of clouds \textit{(on large scale processes)}, in climate models, has proven difficult. \textbf{kilde} This thesis will examine the potential for data driven methods to model cloud behaviour. For this application, the compilation of a new dataset was deemed necessary. Named \acrfull{ecc}, and composed of reanalysis data from ERA5 and satellite retrievals from \acrfull{msg}. With a temporal resolution of one hour and spatial resolution of $0.25^o$ it should be possible to extract suitable features to make predictions, given that the average lifetime of a cloud is an hour \textbf{kilde Lohman}.

%Machine learning methods has shown great performance in other fields such as \textbf{XXX}. In this thesis \textbf{We/I} want to examine the potential of paramerising cloud cover using a machine learning and traditional statistical methods. 
\section{Why data driven learning} \label{sec:intro_deep_learning}
%The methods considered is traditional statistical models (Autoregressive models) and newer deep learning \textit{algorithms} (Convolutional LSTM). Recurrent networks have proven valuable in fields such as speech recognition and image captioning, but will the beat the more used, auto regressive models. 
After decades of being mainly a research area of limited use, today artificial intelligence in the form of deep learning has become a part of our daily life. Face recognition technology has the ability to unlock your phone. Speech recognition allows you to dictate text messages. Self-driving buses are used in metropolitan areas around the world, including Oslo city centre. By manipulating images, you can make Putin to ride a horse. \textbf{cite GAN article} Video manipulation can make appear to say and do things they never did. 

Deep learning has already proven its value in many fields. The scope of this is to implement and compare different methods for data driven learning to find the most suitable method for cloud predictions. For this task autoregressive models and \acrfull{convlstm} are considered. 
%\textit{This thesis attempts to answer if its a appropriate method for solving cloud predictions.} \textit{It remains a open question if its the best model for tackling this task, predicting cloud cover.} %Even thought machine leaning is has proven able for great advances in certain fields. 
Occam's razor is an old philosophical concept. It states that if two hypotheses are equally likely, the simplest one should be preferred. \textbf{kilde (ikke wikipedia)}.

There is currently a huge interest in and activities around machine learning and artificial intelligence.
The practical applications to climate research are under investigation. The last few years researchers have been attempting to incorporate machine learning into geosciences. Attempting to solve a wide range of problems, from rainfall runoff modelling (cite krazerts) to  high-resolution weather forecasting (cite Rodrigues).
% and air quality forecasting (sun and liu). %, precipitaiton nowcasting (Shi et al) and \textbf{kanskje: LES} \textit{deep neural network based feature representation for weather data.} \textbf{lui et al }. %Some attempts was more successful than others. 
Another more comprehensive machine learning project is lead by Tapio Schneider at Caltech. Along with his team of technologist they have ambitions to create a earth system model using machine learning. He and his team of researchers from MIT and former employees of Microsoft and Google hope to create a platform that can resolve clouds and hopefully reduce the uncertainty in climate sensitivity. \textbf{cite Science} 

\textit{The data driven approach hold the potential to extract the correct information, across scales, necessary for predicting cloud cover at a satisfactory precision. Developing methods that eventually can contribute to reducing the spread in climate sensitivity. Producing a more accurate estimate of global mean temperature increase.} \textbf{skriv om}

\textbf{The 80/20 data science dilemma} Legg det til i introduksjonen. Most data scientists spend only 20 percent of their time on actual data analysis and 80 percent of their time finding, cleaning, and reorganizing huge amounts of data.

\section{Structure and Implementations}
The numerical methods used in this thesis are described in Chapter \ref{ch:num_methods}. The code is available on GitHub in the project repository named ``MS'' on \href{https://github.com/hannasv/MS}{https://github.com/hannasv/MS}. 


The code is developed in Python3.7, a popular language for scientific software development. The source code is stored in the package \textit{scloud}, made available on Github trought the project repository. Developed modules draw inspiration from the structure of \textit{scikit-learn} (\cite{sklearn_api}) and the \textit{keras-tuner} (\cite{chollet2015kerastuner}). The \acrshort{convlstm} is implemented in \textit{keras} (\cite{chollet2015keras}) using \textit{tensorflow} as a backend (\cite{tensorflow2015}). The \textit{keras-tuner} is used to automize the hyperparameter search (\cite{chollet2015kerastuner}).

Vizualizations are generated using \textit{Matplotlib} (\cite{matplotlib}), \textit{Seaborn} (\cite{seaborn}) and maps are generated using the package \textit{Cartopy} (\cite{Cartopy}). Other illustrations are developed using TIkZ, a language used for producing technical illustrations within the environment of LaTEX.

%The repository contains everything need to reproduce the results in this study.
%The experiments are conducted in notebooks and the developed modules are stored in the package ``sciclouds''. Descriptions on how to acquire the data (scripts if possible) and project environment is provided to simplify the process.
The versions are documented in the \textit{requirements.txt} and the project environment called ``sciclouds'' is ready for installation. This is a conda environment, the yaml-file lists the Python packages and requirements necessary for running this code. Below you find the code example for installing the environment.

% Included in the readme file on github. 
\begin{verbatim}
git clone https://github.com/hannasv/MS.git
cd MS
conda env create -f environment.yml
conda activate sciclouds
python setup.py install # installing package from source
\end{verbatim}

Supplementary material for remapping satellite data and filtering masks is available in the supplementary repository \href{https://github.com/hannasv/MS-suppl}{https://github.com/hannasv/MS-suppl}. The filters are generated from within the environment of PyAEROCOM (\cite{pyaerocom}). To make use of all the functionality this repo needs to be cloned in the same directory as ``MS''.



