\tdplotsetmaincoords{60}{110}
%
\pgfmathsetmacro{\rvec}{1.6}
\pgfmathsetmacro{\thetavec}{30}
\pgfmathsetmacro{\phivec}{60}

\pgfmathsetmacro{\deltathetavec}{40}
\pgfmathsetmacro{\deltaphivec}{80}

\begin{figure}
    \centering
    
    
\tdplotsetmaincoords{60}{110}
%
\pgfmathsetmacro{\rvec}{1.0}

\pgfmathsetmacro{\thetavec}{30}
\pgfmathsetmacro{\deltathetavec}{40}
\pgfmathsetmacro{\deltatwothetavec}{50}
\pgfmathsetmacro{\deltathreethetavec}{60}

\pgfmathsetmacro{\phivec}{-30}
\pgfmathsetmacro{\deltaphivec}{10}
\pgfmathsetmacro{\deltatwophivec}{50}
\pgfmathsetmacro{\deltathreephivec}{90}

\begin{tikzpicture}[scale=5,tdplot_main_coords]

    %%%%%%%%%%%%%% Setting up axis and coordinate system.
    \coordinate (O) at (0,0,0); % origo
    \coordinate (z) at (0, 0, \rvec); % origo
    %\draw[thin, <->] (1, 0.7, 1.16) -- (1, 0.52, 1.2) node[pos = 0.8, above right]{\Large $d\theta$};
    \draw[very thick,->, opacity = 1.] (0,0,0) -- (1.7, 0, 0) node[anchor=north east]{\Large $x$};
    \draw[very thick,->,  opacity = 1.] (0,0,0) -- (0, 1.7, 0) node[anchor=north west]{\Large $y$};
    \draw[very thick,->,  opacity = 1.] (0,0,0) -- (0, 0, 1.7) node[anchor=south]{\Large $z$};
    \shade[ball color = teal, opacity = 0.1] (0,0,0) circle [radius=\rvec];
    \draw (0,0,0) circle [radius=\rvec];

    %%%%%%%%%%%%%%%%%%%%%%%%%%%% first column
    \tdplotsetcoord{P}{\rvec}{\thetavec}{\phivec}
    \tdplotsetcoord{dP}{\rvec}{\deltathetavec}{\phivec}
    \tdplotsetcoord{G}{\rvec}{\thetavec}{\deltaphivec}
    \tdplotsetcoord{dG}{\rvec}{\deltathetavec}{\deltaphivec}
    
    \draw[dashed, very thick, color=teal, fill = teal, opacity = 0.2] (P) -- (dP) -- (dG) -- (G) -- (P);
    \draw[dashed, very thick, color=teal] (P) -- (dP) -- (dG) -- (G) -- (P);
    
    \tdplotsetcoord{a}{\rvec}{\deltathetavec}{\phivec}
    \tdplotsetcoord{b}{\rvec}{\deltatwothetavec}{\phivec}
    \tdplotsetcoord{c}{\rvec}{\deltathetavec}{\deltaphivec}
    \tdplotsetcoord{d}{\rvec}{\deltatwothetavec}{\deltaphivec}
    
    \draw[dashed, very thick, color=teal, fill = teal, opacity = 0.2] (a) -- (b) -- (d)-- (c) -- (a);
    \draw[dashed, very thick, color=teal] (a) -- (b) -- (d)-- (c) -- (a);
    
    \tdplotsetcoord{e}{\rvec}{\deltatwothetavec}{\phivec}
    \tdplotsetcoord{f}{\rvec}{\deltathreethetavec}{\phivec}
    \tdplotsetcoord{g}{\rvec}{\deltatwothetavec}{\deltaphivec}
    \tdplotsetcoord{h}{\rvec}{\deltathreethetavec}{\deltaphivec}
    
    \draw[dashed, very thick, color=teal, fill = teal, opacity = 0.2] (e) -- (f) -- (h)-- (g) -- (e);
    \draw[dashed, very thick, color=teal] (e) -- (f) -- (h)-- (g) -- (e);

    %%%%%%%%%%%%%%%%%%%%%%%%%%%%%%%% second column
      
    \tdplotsetcoord{P}{\rvec}{\thetavec}{\deltaphivec}
    \tdplotsetcoord{dP}{\rvec}{\deltathetavec}{\deltaphivec}
    \tdplotsetcoord{G}{\rvec}{\thetavec}{\deltatwophivec}
    \tdplotsetcoord{dG}{\rvec}{\deltathetavec}{\deltatwophivec}
    
    \draw[dashed, very thick, color=teal, fill = teal, opacity = 0.2] (P) -- (dP) -- (dG) -- (G) -- (P);
    \draw[dashed, very thick, color=teal] (P) -- (dP) -- (dG) -- (G) -- (P);
  
      
    \tdplotsetcoord{a}{\rvec}{\deltathetavec}{\deltaphivec}
    \tdplotsetcoord{b}{\rvec}{\deltatwothetavec}{\deltaphivec}
    \tdplotsetcoord{c}{\rvec}{\deltathetavec}{\deltatwophivec}
    \tdplotsetcoord{d}{\rvec}{\deltatwothetavec}{\deltatwophivec}
    
    \draw[dashed, very thick, color=teal, fill = teal, opacity = 0.2] (a) -- (b) -- (d)-- (c) -- (a);
    \draw[dashed, very thick, color=teal] (a) -- (b) -- (d)-- (c) -- (a);
  
  
    \tdplotsetcoord{e}{\rvec}{\deltatwothetavec}{\deltaphivec}
    \tdplotsetcoord{f}{\rvec}{\deltathreethetavec}{\deltaphivec}
    \tdplotsetcoord{g}{\rvec}{\deltatwothetavec}{\deltatwophivec}
    \tdplotsetcoord{h}{\rvec}{\deltathreethetavec}{\deltatwophivec}
    
    \draw[dashed, very thick, color=teal, fill = teal, opacity = 0.2] (e) -- (f) -- (h)-- (g) -- (e);
    \draw[dashed, very thick, color=teal] (e) -- (f) -- (h)-- (g) -- (e);
    
    
    %%%%%%%%%%%%%%%%%%%%%%%%%%%%%%%% third column
    \tdplotsetcoord{P}{\rvec}{\thetavec}{\deltatwophivec}
    \tdplotsetcoord{dP}{\rvec}{\deltathetavec}{\deltatwophivec}
    \tdplotsetcoord{G}{\rvec}{\thetavec}{\deltathreephivec}
    \tdplotsetcoord{dG}{\rvec}{\deltathetavec}{\deltathreephivec}
    
    \draw[dashed, very thick, color=teal, fill = teal, opacity = 0.2] (P) -- (dP) -- (dG) -- (G) -- (P);
    \draw[dashed, very thick, color=teal] (P) -- (dP) -- (dG) -- (G) -- (P);
  
    \tdplotsetcoord{a}{\rvec}{\deltathetavec}{\deltatwophivec}
    \tdplotsetcoord{b}{\rvec}{\deltatwothetavec}{\deltatwophivec}
    \tdplotsetcoord{c}{\rvec}{\deltathetavec}{\deltathreephivec}
    \tdplotsetcoord{d}{\rvec}{\deltatwothetavec}{\deltathreephivec}
    
    \draw[dashed, very thick, color=teal, fill = teal, opacity = 0.2] (a) -- (b) -- (d)-- (c) -- (a);
    \draw[dashed, very thick, color=teal] (a) -- (b) -- (d)-- (c) -- (a);
  
    \tdplotsetcoord{e}{\rvec}{\deltatwothetavec}{\deltatwophivec}
    \tdplotsetcoord{f}{\rvec}{\deltathreethetavec}{\deltatwophivec}
    \tdplotsetcoord{g}{\rvec}{\deltatwothetavec}{\deltathreephivec}
    \tdplotsetcoord{h}{\rvec}{\deltathreethetavec}{\deltathreephivec}
    
    \draw[dashed, very thick, color=teal, fill = teal, opacity = 0.2] (e) -- (f) -- (h)-- (g) -- (e);
    \draw[dashed, very thick, color=teal] (e) -- (f) -- (h)-- (g) -- (e);
    
    %%%%%%%%%%%%%%%%%%%%%%%%%% Adding coordinate information

    % First column
    \draw[thick](0.3, -0.25, 0.5)node[scale=0.8, rotate = -15]{$(i,j-1)$};
    \draw[thick](0.3, -0.17, 0.7)node[scale=0.8, rotate = -15]{$(i+1,j-1)$};
    \draw[thick](0.3, -0.3, 0.3)node[scale=0.8, rotate = -15]{$(i-1,j-1)$};

    % Second column
    \draw[thick](0.5, 0.3, 0.65)node[scale=0.8, rotate = 5]{$(i,j)$};
    \draw[thick](0.5, 0.3, 0.85)node[scale=0.8, rotate = 5]{$(i+1,j)$};
    \draw[thick](0.5, 0.3, 0.45)node[scale=0.8, rotate = 5]{$(i-1,j)$};
    
    % Third column
    \draw[thick](0.5, 0.73, 0.87)node[scale=0.8, rotate = 35]{$(i,j+1)$};
    \draw[thick](0.5, 0.8, 0.7)node[scale=0.8, rotate = 35]{$(i-1,j+1)$};

    \draw[thick](0.05, 0.45, 0.73)node[scale=0.8, rotate = 35]{$(i+1,j+1)$};

    \tdplotsetthetaplanecoords{\phivec};
    \shade[ball color=teal,tdplot_screen_coords,opacity=0.1] (O) circle[radius=\rvec];
    \foreach \X/\Y in {xy/z,yz/x,zx/y}
        {\begin{scope}[canvas is \X\space plane at \Y=\rvec]
         \fill circle[radius=1pt];
        \end{scope}}
    \end{tikzpicture}
    
    \caption{Illustrated the relative size of neighbouring pixels (in final uniform grid of ECC) in spherical coordinates. The areas of pixels in a uniform grid decrease poleward. \textbf{Har du en ide om hva denne figuren kan endres til eller om den kan forbedres på noen måte.}}
    
    \label{fig:relative_size_neigbouring_pixels}
\end{figure}