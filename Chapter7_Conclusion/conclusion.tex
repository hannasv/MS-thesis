\chapter{Conclusions and Future Work}
% \textbf{Det du skal svare på fra Introduksjon, The scope of this study is to implement and compare different methods for data driven learning to find the most suitable method for cloud cover predictions.}
In this study a high quality dataset, \acrfull{ecc} was compiled for the purpose of performing data-driven learning of cloud fractional cover. This involves the development of \acrfull{awrs}. Without this the proposed methods are useless.
%The fist objective in this study, was to generate a high quality dataset. 
 
As a ``proof of concept''-study, two models, \acrfull{ar} and \acrfull{convlstm} models were built on the \acrshort{ecc}. They were evaluated and compared against exist parameterizations in ERA5. 
%The second objective was to build, train and evaluate the statistical models. 

On skill they rank 1. AR, 2. ERA5 and 3. ConvLSTM, but on their ability to conserve spatial relations in predicting a sequence, the most important task, they rank 1. ERA5, 2. AR, 3. ConvLSTM.
For the first few steps of the predicted sequence, the AR models does well, onward the performance degrade.

\textbf{List the skill of the models.}
This study shows that model X performed best. $ConvLSTM-B_{10}-SL_{24}-32-3\times3-32-3 \times3$-model was the best architecture, for \acrshort{ar}-models it was ..

In conclusion, at their current state they are not appropriate for implementations toward practical applications. However, with a few more experiments and more refined tuning they have shown the potential for applying data driven learning fro the cloud forecasting problem and other similar problem in Geosciences. 
This would evolve using automatic hyperparameter tuning. This study has learned which combinations of number of layers, hidden states and kernels to use to circumvent memory issue. Exploring the effect of altering many of the parameters kept constant would be of interest.

Followed by a evaluation of the developed parameterizations in the context of a full climate model. Using the output from the ensembles in CMIP6 and see how much the cloud cover changes. A reasonably accurate model can be useful inn practical application without scoring perfectly on the metrics, here \acrshort{mae}.

For applications to climate science it common to predict centuries not day and night as we did in this study. Invest more effort in predicting longer sequences. Climate models usually make projections hundreds of years into the future.

\textbf{Fra INa: Makinlæring i seg selv en kanskje ikke godt nok for å beskrive kompleske prosesser som maskinlæring, men at komplekse prosesser som skydannelse i kombinasjon kan gi gode resultater. At fysikken beskriver prinsippene og mskinlæring korrigerer prinsippene mot observasjoner. Kombinasjonen er løsningen.}

%Unfortunately, the best model is not sufficiently accurate to predict cloud cover. 
%For future work it would be interesting to investigate how ..?
%demonstrate its advantage and outperform ..?
% The data driven approach taken in this thesis does not net this case-specific adjustment. Doesn't need different approaches for different regimes, but relies on the satellites capabilities to detect them. Cirrus being the most difficult. 

\section{Questions for Trude}
\begin{enumerate}
    \item Comment from Hugo, Name of chapter 2, theoretical background implies that it contains the numerical methods as well. How about changing the to Climate and Cloud Physics?
    \item Riktig skrevet: This heat gets trapped in the earth system,?
    \item sjekker du siste avsnitt i Abstract å ser om jeg har skrevet noe hårreisende .. ?
    \item Må jeg skrive en oversikt over de forskjellige reanalysene eller er det tilstrekkelig, å si at jeg velger ERA5 fordi den har bedre romlig oppløsning enn JRA og MERRA, som er de to andre naturnlige valge. Pluss at den er ny så det er lite forskning og er det er mer spennende med resultater i den retningen?
    \item \textbf{Litt usikker på inndeling i seksjoner, blir noen paragraphs etter jeg slo sammen }
\end{enumerate}