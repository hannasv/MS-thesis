\chapter{Conclusions}
% \textbf{Det du skal svare på fra Introduksjon, The scope of this study is to implement and compare different methods for data driven learning to find the most suitable method for cloud cover predictions.}
In this study a high quality dataset, \acrfull{ecc} was compiled for the purpose of performing data-driven learning on cloud fractional cover. This involves the development of \acrfull{awrs}. %, necessary to regrid satellite data to cloud fractional cover.
Two models, \acrfull{ar} and \acrfull{convlstm} models were built and evaluated based on their ability to predict a sequence of cloud cover. This was compared to the data in ERA5. 

The fist objective in this study, was to generate a high quality dataset. Without this the proposed methods are useless. The second objective was to build, train and evaluate the statistical models. 
%Alternative, at its current state, inhomogenities renders, \acrshort{ecc}, inappropriate for cloud cover forecasting. This is not the same as stating its of low quality, however its not suitable for the applications in this study. 

%If the results are bad, the conclusion is that the dataset is not of high enough quality. This is not uncommon when working with cloud data. Don't make it look like you took bad choices when building the dataset it may simply mean that the nature of the problem is difficult. Which it is..
$ConvLSTM-B_{10}-SL_{24}-32-3\times3-32-3 \times3$ 
This study shows that model X performed best. The models where able to learn, however dataset, \acrshort{ecc}, is not of high enough quality and they couldn't compete with ERA5.

Spending more time tuning the \acrshort{convlstm}-model, expesiaclly the optimizer and changing the weight init scheme to xavier according to paper in download its important. 

In the case of bad model performance. In conclusion, a higher quality dataset is necessary to apply these sets of models to the cloud forecasting problem. The models are state-of-the-art 

Achivement, mangaed to train a model on a larger dataset than the previos studies.

\section{Future work}
A reasonable accurate model, may prove to have a practical application even though the \acrshort{mae} is imperfect. Evaluate the parameterisation in the context of a climate model. Using the output of the relevant variables from a \acrshort{ESM} ... \textbf{Trude they all have these variables right?}

Based on the small set of experiments run in this section, there is 
Datasets publish have different versions. Propose a suggestion for the content on \acrshort{ecc} v2.

Invest more effort in predicting longer sequences. Climate models usually make projections hundreds of years into the future.

Sensitivitets test på input variablene.

%Unfortunately, the best model is not sufficiently accurate to predict cloud cover. 
%For future work it would be interesting to investigate how ..?
%demonstrate its advantage and outperform ..?

Nå som vi har funnet ut hvilke kombinasjoner av hidden states, filter, batches og sequence length vi må begrense oss til hadde det vært interresant å bruke keras-tuner for å automatisere optimizeringen av noen av parameterne som ble hold konstante i denne studien, vekt init, optimizer and learning rates, ... Finn flere?

% The data driven approach taken in this thesis does not net this case-specific adjustment. Doesn't need different approaches for different regimes, but relies on the satellites capabilities to detect them. Cirrus being the most difficult. 

\section{Questions for Trude}

\begin{enumerate}
    \item Comment from Hugo, Name of chapter 2, theoretical background implies that it contains the numerical methods as well. How about changing the to Climate and Cloud Physics?
    \item Riktig skrevet: This heat gets trapped in the earth system,?
    \item sjekker du siste avsnitt i Abstract å ser om jeg har skrevet noe hårreisende .. ?
    \item Må jeg skrive en oversikt over de forskjellige reanalysene eller er det tilstrekkelig, å si at jeg velger ERA5 fordi den har bedre romlig oppløsning enn JRA og MERRA, som er de to andre naturnlige valge. Pluss at den er ny så det er lite forskning og er det er mer spennende med resultater i den retningen?
    \item \textbf{Litt usikker på inndeling i seksjoner, blir noen paragraphs etter jeg slo sammen }
\end{enumerate}