\chapter{Conclusions}

\section{Summary of contributions and main findings }
State something great about the dataset.

This study shows that model X performed best. However, since such few experiments was conducted it is important to underline the potential of further development of this models. Both efficiency for regression case, and automatic hyperparameter tuning for the machine learning models. Using the keras tuner, it is made available for the reader in the project GitHub repo. It is not performed in this study due to time limitiation and limited computational resources. But here lies great potential to find the best machine learning model. 

Another area, regulationg the flow of data to optimalize the architectures possible.

Using Tensorflow the system easily adapts to other devises, making it easy to run on other clusteres as well. Removing the required knowledge of the user about distributed systems. 



\section{Future work}

Most important. Investigate model configurations to improve performance on longer sequences -- climate models typically predicts a century.

In future work it would be interesting to asses how data driven parametrisation compare to the existing parametrisaions available in the state of the art climate models. Here both the temporal and spatial resolution is a lot coarser. Other data sets could be considered. The masks in other data sets are computed based on more channels than in METeosat but the temporal resolution is a lot worse. 



\section{Final remark}
Text