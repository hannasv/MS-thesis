\chapter{Conclusions}
% \textbf{Det du skal svare på fra Introduksjon, The scope of this study is to implement and compare different methods for data driven learning to find the most suitable method for cloud cover predictions.}
In this study a high quality dataset, \acrfull{ecc} was compiled for the purpose of performing data-driven learning on cloud fractional cover. This involves the development of \acrfull{awrs}. %, necessary to regrid satellite data to cloud fractional cover.
Two models, \acrfull{ar} and \acrfull{convlstm} models were built on the \acrshort{ecc}. Followed by an evaluation of their ability to predict a sequence of cloud cover and comprison to the existing parameterization in ERA5. 

The fist objective in this study, was to generate a high quality dataset. Without this the proposed methods are useless. The second objective was to build, train and evaluate the statistical models. 
The best  

This study shows that model X performed best. $ConvLSTM-B_{10}-SL_{24}-32-3\times3-32-3 \times3$-model was the best architecture 


In conclusion, this was a ``proof of concept'' study, proving that the statistical models chosen for this were able to learn. Based on the example sequences shown, the show a degraded performance for lager predictions step. In their current state they are not available for implemetations in practical applications. However, with a few more experiments and more refined tuning they have shown the potensial for appliyng data driven learning fro the cloud forecasting problem and other similar problem in geosciences. 

\textbf{Legg til andre områder du kan forbedre. }


\section{Future work}

After a few additional experiments with refined tuning it would be interesting to evaluate the model in the context of a full climate model. Using the output from the ensamebles in CMIP6 and se how much the cloud cover differes?

Nå som vi har funnet ut hvilke kombinasjoner av hidden states, filter, batches og sequence length vi må begrense oss til hadde det vært interresant å bruke keras-tuner for å automatisere optimizeringen av noen av parameterne som ble hold konstante i denne studien, vekt init, optimizer and learning rates, ... Finn flere?

A reasonable accurate model, may prove to have a practical application even though the \acrshort{mae} is imperfect. Evaluate the parameterisation in the context of a climate model. Using the output of the relevant variables from a \acrshort{ESM} ... 

It would be interresting to perform sesitivity analysis on the input variables to determine is all four envionmental variables are necessary or if a subset of them would be suffifienct. 

For applications to climate sciences the parameterization need to be able to predict sequences in the order of centuries. \textbf{ar}-modellen kan dette. Det kan vell ConvLSTM også siden inputenen er variabler som for et tidssteg kan produseres av \acrshort{esm} prior to the diagnisation of cloud cover. 

Invest more effort in predicting longer sequences. Climate models usually make projections hundreds of years into the future.

%Unfortunately, the best model is not sufficiently accurate to predict cloud cover. 
%For future work it would be interesting to investigate how ..?
%demonstrate its advantage and outperform ..?



% The data driven approach taken in this thesis does not net this case-specific adjustment. Doesn't need different approaches for different regimes, but relies on the satellites capabilities to detect them. Cirrus being the most difficult. 

\section{Questions for Trude}

\begin{enumerate}
    \item Comment from Hugo, Name of chapter 2, theoretical background implies that it contains the numerical methods as well. How about changing the to Climate and Cloud Physics?
    \item Riktig skrevet: This heat gets trapped in the earth system,?
    \item sjekker du siste avsnitt i Abstract å ser om jeg har skrevet noe hårreisende .. ?
    \item Må jeg skrive en oversikt over de forskjellige reanalysene eller er det tilstrekkelig, å si at jeg velger ERA5 fordi den har bedre romlig oppløsning enn JRA og MERRA, som er de to andre naturnlige valge. Pluss at den er ny så det er lite forskning og er det er mer spennende med resultater i den retningen?
    \item \textbf{Litt usikker på inndeling i seksjoner, blir noen paragraphs etter jeg slo sammen }
\end{enumerate}