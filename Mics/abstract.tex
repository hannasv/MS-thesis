\chapter*{Abstract}
\addcontentsline{toc}{chapter}{Abstract}
%\textbf{første setning .. store usikkerhet og at det er knyttet tili klimamodeller og igjen hvorfor det er viktig. Hvorfor er det viktig med korrekte klimamodeller. }
%\textbf{kombine 2 førte }
%Over the last decades researchers have been working on determining the climate sensitivity. Clouds are important in today's climate, reflecting $47W/m^{-2}$ of the solar radiation back into space. 
%Society needs a more accurate description of what may happen in future climates 
%(\cite{Wild2019TheModels}).

%\subsection*{Introduce a challenge or unresolved issue that you will try and solve.} 
Over the last decades researchers have been working on determining the climate sensitivity. 
There is large uncertainties in estimates climate sensitivity for climate projections. This is related to cloud feedbacks and it is attributed to how clouds are resolved in climate models. 
This inspire new parameterizations of cloud cover. It is unclear which level of sophistication is needed for these sub-grid scale parameterizations in order to model their effect on climate.

To investigate this problem from a new angle, namely the potential for data driven learning for parameterization of clouds, a new dataset, \acrfull{ecc}, was generated. This is based on a combination of reanalysis and satellite data. % \textbf{for/in this study}. \textbf{enda tydligere på at jeg har gjort dette}.
Statistical methods, \acrfull{ar}- and \acrfull{convlstm}-models were compared existing parameterizations in \textbf{reanalysis (ECMWF IFS cycle X)}, on their ability to parameterize cloud cover. 

Measured on their skill predicting the period
%The \acrfull{mae} summed over the period 
from 2014 including 2018 the best \acrshort{ar}-model, $AR-B-S-L_5$, had a \acrshort{mae} of X,  the $ConvLSTM-B_{10}-SL_{24}-32-3\times3-32-3 \times3$ had $1.91520128 \cdot 10^8$ and the reanalysis had a skill of $1.16508643\cdot10^8$. Of the models developed in this study the \textbf{1} outperform the \textbf{2}.

In this ``proof of concept'' study it was proven that statistical \textbf{name best model} models was able to learn how to forecast cloud cover. Unfortunately, it didn't learn enough \textbf{based on four enviornmental variables} to outperform the state-of-the-art parameterization in reanalysis (ERA5). Hopefully this study can serve as motivation and a suitable starting point for similar studies. 
%\textbf{This was a proof of concept study, we proved that the statistical models was able to learn. However they were not sufficiently accurate to outperform the cloud parameterisation in reanalysis. }
%\textbf{selvom modellen lærte noe, lærte den ikke nok... }
%Unfortunately, they were unable to compete with the reanalysis.
%Autoregressive (\acrshort{ar}) outperform the more complex \acrshort{convlstm}-model. 
% List the results in Table X
%\textbf{Denne studien har bidrat til å skaer en forståelse for at det ...selv med høuy opppløsning er det ikke nødvendig vis godt nok for å  }
%In its current state the dataset is not of high enough quality to outperform the reanalysis on the task of predicting cloud cover. 
\cleardoublepage