\chapter*{Abstract}
\addcontentsline{toc}{chapter}{Abstract}
Ikke skriv masteroppgaven som om LSTM kommer til å funke. Skriv generelt at i denne oppgaven skal vi test to metoder. Hugo. 
Skriv også om alle valg som er tatt i hensyn til når du lager datasettet. Dette er viktigere enn sykt mye teoridel. 


\subsection*{Introduce topic and why its important}
Over the last decades researchers have been working on determining the climate sensitivity. The fifth assessment report result in a range of 1.5K to 4.7K. Here 4.7K allows for a doubling of the $CO_2$ emissions compared to 1.5K. Most of this uncertainty is attributed to how clouds are resolved in climate models. (Her har jeg vell sagt motsatt ting.) We known clouds are important in today's climate. They reflect 50\% of the solar radiation back into space. %Trap less heat because of global warming. 

\subsection*{Introduce a challenge or unresolved issue that you will try and solve.} 
Large uncertainties associated with the effect of clouds in future climates. Its unclear to which level of sophistication these sub-grid scale parametrizations need to be in order model their effect on climate.

\subsection*{What have you done to try and solving this.}
In order to investigate the data driven learning potensial for parameterisation of clouds, a new dataset (European Cloud Cover) was compiled. Methods like traditional statistical models and ConvLSTM was used to relate simple meterorological surface variables to cloud cover. The sophistication of a parameterisation is never know a-priori.

\subsection*{Main result - Include the numerical result of your best model.}

\subsection*{The implications in the context of 1+2.}
\cleardoublepage