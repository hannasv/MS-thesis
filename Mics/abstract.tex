\chapter*{Abstract}
\addcontentsline{toc}{chapter}{Abstract}
%\textbf{første setning .. store usikkerhet og at det er knyttet tili klimamodeller og igjen hvorfor det er viktig. Hvorfor er det viktig med korrekte klimamodeller. }
%\textbf{kombine 2 førte }
%Over the last decades researchers have been working on determining the climate sensitivity. Clouds are important in today's climate, reflecting $47W/m^{-2}$ of the solar radiation back into space. 
%Society needs a more accurate description of what may happen in future climates 
%(\cite{Wild2019TheModels}).

%\subsection*{Introduce a challenge or unresolved issue that you will try and solve.} 
Over the last decades researchers have been working on determining the climate sensitivity. 
There is large uncertainties in estimates climate sensitivity for climate projections. This is related to cloud feedbacks and it is attributed to how clouds are resolved in climate models. 
This inspire new parameterizations of cloud cover. It is unclear which level of sophistication is needed for these sub-grid scale parameterizations in order to model their effect on climate.

To investigate this problem from a new angle, namely the potential for data driven learning for parameterization of clouds, a new dataset, \acrfull{ecc}, was generated. This is based on a combination of reanalysis and satellite data. % \textbf{for/in this study}. \textbf{enda tydligere på at jeg har gjort dette}.
Statistical methods, \acrfull{ar}- and \acrfull{convlstm}-models were compared existing parameterizations in \textbf{reanalysis (ECMWF IFS cycle X)}, on their ability to parameterize cloud cover. 

Measured on their skill predicting the period
%The \acrfull{mae} summed over the period 
% 0.20386150843063067 & 0.04901 &  0.3362179514756918
from 2014 including 2018 the best \acrshort{ar}-model, $AR-B-L5$-model, had a \acrshort{mae} of $0.04901$,  the $ConvLSTM-B_{10}-SL_{24}-32-3\times3-32-3 \times3$ had $0.33621$ and the reanalysis had a skill of $0.20386$. Measured on \acrshort{mae}, the $AR-B-L5$-model outperform the $ConvLSTM-B_{10}-SL_{24}-32-3\times3-32-3 \times3$-model. On the other hand when it comes to capturing devlopments in a sequence the result is oposite. 

Explain by the optimization routine, \acrshort{ar} is optimized to fit the next timestep and \acrshort{convlstm} is optimized to fit a 24 hour sequence.

In this ``proof of concept'' study it was proven that $AR-B-L5$- and $ConvLSTM-B_{10}-SL_{24}-32-3\times3-32-3 \times3$-model was able to learn how to forecast cloud cover. Unfortunately, the models have somthing more left to learn before they can compete with state-of-the-art parameterization feks. in reanalysis (ERA5). Hopefully this study can serve as motivation and a suitable starting point for similar studies.  
\cleardoublepage