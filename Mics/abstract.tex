\chapter*{Abstract}
\addcontentsline{toc}{chapter}{Abstract}
%\textbf{Pass på at plassen et tema får i abstract og oppgaven generelt er representativut på hvor mye tid du har brukt på det...}

%\subsection*{Introduce topic and why its important}
Over the last decades researchers have been working on determining the climate sensitivity. %The fifth assessment report result in a range of 1.5K to 4.7K. Here 4.7K allows for a doubling of the $CO_2$ emissions compared to 1.5K. 
Clouds are important in today's climate, reflecting $47W/m^{-2}$ (\cite{Wild2019TheModels}) of the solar radiation back into space. Society needs a more accurate description of what may happen in future climates.
%Trap less heat because of global warming.
%TS:Det er ikke riktig at skyer reflekterer 50% - de reflekterer 50W/m2, som er mye mindre enn 50%

%\subsection*{Introduce a challenge or unresolved issue that you will try and solve.} 
There is large uncertainties in estimates climate sensitivity for climate projections. This is related to cloud feedbacks and it is attributed to how clouds are resolved in climate models. %\textbf{Du har ikke sagt noe om "uncertainty" så langt, så her må det først legges inn en setning om at mye av usikkerheten i klimafølsomhet og dermed klima-fremsrivninger skyldes sky-tilbakekoplinger}
%TS Du har ikke sagt noe om "uncertainty" så langt, så her må det først legges inn en setning om at mye av usikkerheten i klimafølsomhet og dermed klima-fremsrivninger skyldes sky-tilbakekoplinger 
%Large uncertainties associated with the effect of clouds in future climates. 
It is unclear which level of sophistication is needed for these sub-grid scale parameterizations in order to model their effect on climate.

%\subsection*{What have you done to try and solving this.}
To investigate this problem from a new angle,  namely the potential for data driven learning for parameterization of clouds, a new dataset, \acrfull{ecc}, was compiled based on a combination of reanalysis and satellite data. 
%TS: Nei, dette holder i et abstract, men du kan kanskje si at det er basert på satellitdata

Statistical methods, \acrfull{ar}- and \acrfull{convlstm}-models were compared against ERA5, on their ability to parameterize cloud cover.

%\subsection*{Main result - Include the numerical result of your best model.}
The \textbf{name best AR} had a performance of \textbf{X} and the \textbf{name architecture of best convlstm} had a performance of \textbf{Y}. Autoregressive (\acrshort{ar}) outperform the more complex  \acrshort{convlstm}-model. Unfortunately, these models performance can't compete with the parameterizations available in ERA5, having a score of \textbf{X}. 
%\textbf{Går dette under implications}
%In summary this study is the produced dataset, applied it to numerical experiment performing the task of forecasting cloud cover. 
In its current state the dataset is not of high enough quality to outperform ERA5 on the task of predicting cloud cover. Hopefully this can serve as motivation and a suitable starting point for similar studies. 

%\subsection*{The implications in the context of 1+2.}
%ConvLSTM demonstrate is ability to detect spatial relations, and offer a staring point for other studies.
%development..? \textbf{How to include the implications of what seems to be two bad models.}

\cleardoublepage