\section{Practical implications} \label{sec:practical_implications}
%It is necessary to have a understanding of the needs of the end product before conducting large machine learning projects. Answering questions like: What will it be used for and how can it be implemented in useful way?

A major downside of the data driven learning approach is the rigid resolution. A trained model can only be used on similar problems, with the same spatiotemporal resolution. For applications like climate models, output comes in a wide range of different resolutions. Before implementing the finished product in a new model of a different resolution, it would need to be retrained on the resolution of the climate model under development. This process involves both remapping of the dataset and retraining the model at the correct resolution. This is a time consuming process involving finding a new set of hyperparameters suitable for the new resolution. % It essentially means starting over.

Once trained on global climate datasets, machine learning models provide fast results even for complex parameterization which is what makes them suitable for the application of climate modelling. Most machine learning packages are developed using Python. \acrfull{esm} are implemented in python. Methods for including the trained parameterizations need to be developed.
 